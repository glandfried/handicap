\documentclass[shownotes,aspectratio=169]{beamer}
 \mode<presentation>
 {
  \usetheme{Antibes}
  \setbeamertemplate{navigation symbols}{}
 }
 
\input{auxiliar/tex/diapo_encabezado.tex}
\usepackage{todonotes}% INCOMPATIBLE con Tikz

%\input{./auxiliar/tex/tikzlibrarybayesnet.code.tex}

%\title[Handicap]{Handicap}

%\author[Gustavo Landfried]{Gustavo Landfried \\ \vspace{0.2cm}
%\scriptsize Lic. Ciencias Antropol\'ogicas. \\
%Doctorando Ciencias de la Computaci\'on \\
%\vspace{-0.3cm}}
%\institute[DC-ICC-CONICET]{Departamento de Ciencias de la Computaci\'on (UBA -- CONICET) \vspace{-0.3cm}}
%\date{}


\begin{document}
\color{black!85}
\small
%\setbeamercolor{background canvas}{bg=gray!8}


\begin{frame}[plain]
\begin{center}
 \Huge Habilidad en Go
\end{center}
\end{frame}

\begin{frame}[plain]
 \centering
 \LARGE Introducci\'on
 
\end{frame}

\begin{frame}[plain]
\begin{textblock}{160}(0,0)
\begin{center}
 \Large Problema 1 (Handicap y Komi)
\end{center}
\end{textblock}
 \vspace{0.75cm} \pause
 
  El juego de Go se caracteriza por tener reglas sencillas pero muchos estados posibles, lo que lo hace muy dif\'icil de aprender.
  
  \vspace{0.3cm} \pause
  
  Por este motivo la diferencia de habilidad en una poblaci\'on de jugadores habituales de Go generalmente es muy grande.
  
 \vspace{0.3cm} \pause
 
  Es una pr\'actica com\'un en partidas de Go amateur otorgar ventaja al jugador menos h\'abil de modo de lograr partidas un algo m\'as parejas. [La mayor cantidad de gente es amateur]
 
 \vspace{0.3cm} \pause
 
  Para elegir la ventaja correcta es necesario contar con buenas estimaciones de habilidad.
  
 \vspace{0.3cm} \pause
  
  Debido a que las competencias profesionales de Go se juegan sin ventaja, el algoritmos estado del arte para estimar de habilidad en Go (WHR) no considera el efecto del handicap.
 
 \vspace{0.3cm} \pause
 
  Esto es un problema, por ejemplo, para la Asociaci\'on Argentina de Go, que se ve obligada a usar un algoritmo alternativo (AGA) que no tiene la calidad necesaria.
  
 \vspace{0.3cm} \pause
 
  Para solucionar este problema, incorporamos handicap y komi al estimador de la industria del video juego (TTT), ofreciendo resultados que superan el nivel predictivo de WHR y AGA. 
 
\end{frame}

\begin{frame}[plain]
\begin{textblock}{160}(0,0)
\begin{center}
 \Large Resultado intuitivo motivador \\ \large Secci\'on metodolog\'ia
\end{center}
\end{textblock}

\todo[inline, color=black!16]{\textbf{Figura}: Motivar la lectura (el inter\'es sobre el algoritmo),  mostrando que si selecciono las partidas que yo considero que tuvieron un sub-asignaci\'on de handicap, entonces la cantidad de partidas a favor del blanco aumenta}
 
\end{frame}

\begin{frame}[plain]
 \centering
 \LARGE Metodolog\'ia
 
\end{frame}


\begin{frame}[plain]
\begin{textblock}{160}(0,0)
\begin{center}
 \Large Base de datos
\end{center}
\end{textblock}

\begin{table}
\begin{tabular}{|c|c|c|c|} \hline
 Base & Tama\~no & \% handicap & Gana blanco \\ \hline
 OGS & $400$ mil &  & 0.55 \\ \hline
 AAGo & $3.5$ mil &  & 0.62 \\ \hline
 KGS &  & & \\ \hline
\end{tabular}
\end{table}


\end{frame}



\begin{frame}[plain]
\begin{textblock}{160}(0,0)
\begin{center}
 \Large Modelo básico
\end{center}
\end{textblock}

 \begin{figure}[H]     
     \centering
     \begin{subfigure}[b]{0.75\textwidth}
       \includegraphics[width=\textwidth]{graph/trueskill.pdf} 
     \end{subfigure}
   \end{figure} 
  
\end{frame}

\begin{frame}[plain]
\begin{textblock}{160}(0,0)
\begin{center}
 \Large Modelo con handicap
\end{center}
\end{textblock}

 \begin{figure}[H]     
     \centering
     \begin{subfigure}[b]{0.75\textwidth}
       \includegraphics[width=\textwidth]{graph/truehandicap.pdf} 
     \end{subfigure}
   \end{figure} 
  
\end{frame}


% \begin{frame}[plain]
% \begin{textblock}{160}(0,0)
% \begin{center}
%  \Large Interpretabilidad del valor de habilidad \\ \large Secci\'on metodolog\'ia
% \end{center}
% \end{textblock}
% 
% \todo[inline, color=black!16]{\textbf{Figura (te\'orica)}, mostrar como cambia la probabilidad de ganar en funci\'on de la diferencia de habilidad (sin y con incertidumbre, y con la diferencia medida en escalas)}
% 
% \end{frame}


\begin{frame}[plain]
\begin{textblock}{160}(0,0)
\begin{center}
 \Large Efecto del h\'andicap
\end{center}
\end{textblock}

   \begin{figure}[H]     
     \centering
     \begin{subfigure}[b]{0.75\textwidth}
       \includegraphics[width=\textwidth]{../figures/ogs/ogs_estimado_handicap.pdf} 
     \end{subfigure}
   \end{figure} 
   
   \pause
   
   \centering
   
   Parece que hay relación lineal

\end{frame}

\begin{frame}[plain]
\begin{textblock}{160}(0,0)
\begin{center}
 \Large Modelo regresión
\end{center}
\end{textblock}

Un único factor multiplicado por el valor del handicap. 

Necesitamos un modelo de pesos.

\end{frame}




\begin{frame}[plain]
\begin{textblock}{160}(0,0)
\begin{center}
 \Large Modelo pesos
\end{center}
\end{textblock}

   \begin{figure}[H]     
     \centering
     \begin{subfigure}[b]{0.75\textwidth}
       \includegraphics[width=\textwidth]{graph/weight.pdf} 
     \end{subfigure}
   \end{figure} 

\end{frame}


\begin{frame}[plain]
\begin{textblock}{160}(0,0)
\begin{center}
 \Large Komi 
\end{center}
\end{textblock}

  \begin{figure}[H]     
     \centering
     \begin{subfigure}[b]{0.5\textwidth}
       \includegraphics[width=\textwidth]{../figures/ogs/komi.pdf} 
     \end{subfigure}
   \end{figure} 
\pause

La regresión es importante sobre todo en Komi que hay muchos valores distintos de komi

\end{frame}


\begin{frame}[plain]
\begin{textblock}{160}(0,0)
\begin{center}
 \Large Komi (factor por segundo)
\end{center}
\end{textblock}

h: handicap

k: komi

r: regresión

\begin{table}
\begin{tabular}{|c|c|c|} \hline
  & \color{blue}{h} & hr\\ \hline
 k  & \color{orange}{h-k} &  \cancel{hr-k} \\ \hline
 kr &  \color{red}{h-kr} & \color{green}{hr-kr} \\ \hline
\end{tabular}
\end{table}


\end{frame}


\begin{frame}[plain]
\begin{textblock}{160}(0,0)
\begin{center}
 \Large Handicap con komi
\end{center}
\end{textblock}

\begin{figure}[H]     
     \centering
     \begin{subfigure}[b]{0.5\textwidth}
       \includegraphics[width=\textwidth]{../figures/ogs/ogs_estimado_handicap_con_komi.pdf} 
     \end{subfigure}
   \end{figure} 

\end{frame}


\begin{frame}[plain]
\begin{textblock}{160}(0,0)
\begin{center}
 \Large Komi
\end{center}
\end{textblock}

\begin{figure}[H]     
     \centering
     \begin{subfigure}[b]{0.5\textwidth}
       \includegraphics[width=\textwidth]{../figures/ogs/ogs_estimado_komi.pdf} 
     \end{subfigure}
   \end{figure} 

\end{frame}


\begin{frame}[plain]
\begin{textblock}{160}(0,0)
\begin{center}
 \Large La predicci\'on a priori (evidencia) como fuente de selecci\'on \\ \large Secci\'on metodolog\'ia
\end{center}
\end{textblock}

\todo[inline, color=black!16]{\textbf{Figuras} para explicar la evidencia. Vamos a tener que seleccionar entre: A) diferentes soluciones propias (komi) validar que tenga sentido la modificaci\'on que pronemos (evidencia con y son facores handiico y/o komi), B) Entre diferentes algoritmos }

\end{frame}


\begin{frame}[plain]
\begin{textblock}{160}(0,0)
\begin{center}
 \Large Komi \\ \large Secci\'on resultados
\end{center}
\end{textblock}

\todo[inline, color=black!16]{\textbf{Tabla} de evidencias comparando opciones. \textbf{Figura} si es necesario}

\end{frame}



\begin{frame}[plain]
\begin{textblock}{160}(0,0)
\begin{center}
 \Large Algoritmos \\ \large Secci\'on resultados
\end{center}
\end{textblock}

\todo[inline, color=black!16]{\textbf{Tabla} comparaci\'on de evidencia respecto de otros modelos comparados usando handicap y komi}

\end{frame}


\begin{frame}[plain]
\begin{textblock}{160}(0,0)
\begin{center}
 \Large El h\'andicap como unidad de medida
\end{center}
\end{textblock}

\todo[inline, color=black!16]{\textbf{Figura} Explicar como reesclar las habilidades para elegir el $\beta$ }

\end{frame}


% 
% \begin{frame}[plain]
%  \centering
%  \LARGE Resultados
%  
% \end{frame}
% 
% 
% \begin{frame}[plain]
% \begin{textblock}{160}(0,0)
% \begin{center}
%  \Large Validaci\'on \\ \large Secci\'on resultados
% \end{center}
% \end{textblock}
% 
%  \todo[inline, color=black!16]{\textbf{Figura de validaci\'on} Diferencia de habilidad vs probabilidad emp\'irica de ganar. Mostrar que la curva emp\'irica, dada la incertidumbre subyacente de las estimaciones, se ajusta a la curva te\'orica con misma incertidumbre.}
%  
% \end{frame}

\begin{frame}[plain]
\begin{textblock}{160}(0,0)
\begin{center}
 \Large Arte (curvas de aprendizaje) \\ \large Secci\'on resultados
\end{center}
\end{textblock}

\todo[inline, color=black!16]{\textbf{Figura art\'istica} Mostrar una selecci\'on de curvas de aprendizaje para mostrar como lucen.}

\end{frame}
% 
% \begin{frame}[plain]
% \begin{textblock}{160}(0,0)
% \begin{center}
%  \Large Estimaciones handicap \\ \large Secci\'on resultados
% \end{center}
% \end{textblock}
% 
% \todo[inline, color=black!16]{\textbf{Figura} Mostrar estimaciones finales (usando tau 0 no va a haber historia)}
% 
% \end{frame}

\begin{frame}[plain]
\begin{textblock}{160}(0,0)
\begin{center}
 \Large Arte y mapeo a tradici\'on \\ \large Secci\'on resultados
\end{center}
\end{textblock}

\todo[inline, color=black!16]{\textbf{Figura 1} Curvas de aprendizaje re-escaladas, \textbf{Figura 2} Curvas de aprendizaje en escalas tradiciones}

\end{frame}


\begin{frame}[plain]
\begin{textblock}{160}(0,0)
\begin{center}
 \Large Problema 2 (Convergencia y Comparabilidad)
\end{center}
\end{textblock}
 \vspace{0.75cm} \pause
 
  La base de datos de la AAGo es muy chica, lo que limita la información disponible para estimar las habilidades (convergencia).
  
 \vspace{0.3cm} \pause
 
 Además, las estimaciones hechas al interior de una base de datos la AAGo no son comparables con las estimaciones obenidas en otras bases de datos (comparabilidad).
 
 \vspace{0.3cm} \pause
 
 El estimador utilizado (TTT), al propagar la información por toda la red de partidas, permite solucionar ambos problemas mediante la identificación de jugadores que participen en más de una base de datos.
 
 \vspace{0.3cm} \pause
 
 Para implementar esta solución, hicimos una encuesta en la AAG0, y obtuvimos respuesta de 28 personas, que ofrecieron sus identificadores en las bases de datos OGS y KGS.
 
 \vspace{0.3cm} \pause
 
 Las estimaciones realizadas en esta meta-base de datos permitirán mejorar sustancialmente las estimaciones de la AAGo y hacer comparables las estimaciones entre las diferenres bases de datos.
 
\end{frame}



\begin{frame}[plain]
 \centering
 \LARGE Discusiones / Conlcusiones
 
\end{frame}


\begin{frame}[plain]
 
 \begin{figure}[H]     
     \centering
     \begin{subfigure}[b]{0.33\textwidth}
       \includegraphics[width=\textwidth]{auxiliar/images/pachacuteckoricancha.jpg} 
     \end{subfigure}
   \end{figure} 
  
\end{frame}


\end{document}



