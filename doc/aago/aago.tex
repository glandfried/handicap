\documentclass[a4paper,10pt]{article}
\usepackage[utf8]{inputenc}
\input{../aux/tex/encabezado.tex}
\input{../aux/tex/tikzlibrarybayesnet.code.tex}

\usepackage{paracol}
%opening

\title{AAGo: Reporte t\'ecnico}
\author{Gustavo Landfried}
\date{\today}

\begin{document}

\maketitle

\begin{abstract}
Se reporta una primera comparaci\'on entre las estimaciones de AAGo y las que surgen del algoritmo Trueskill Through Time (TTT). 
Si bien existen diferencias significativas, estas no ofrecen evidencia respecto de qu\'e algoritmo funciona mejor.
Para evaluarlos solo falta contar con las predicciones a priori de AAGo. 
\end{abstract}

Realizamos estimaciones de habilidad a trav\'es del algoritmo TrueSkill Through Time (Fig~\ref{aago_ttt_escalado}).
La escala de TTT no tiene significado en t\'erminos absolutos.
Lo que importa es la diferencia de habilidad, a partir de la cual se puede calcular la probabilidad de ganar.
Un tarea pendiente consiste en re-escalar los valores absolutos de TTT para que cada punto de habilidad represente el aporte de una piedra de handicap.
Como primera aproximaci\'on para poder comparar las estimaciones de TTT y las la AAGo (Fig~\ref{aago_aga}), lo que se hizo fue re-escalar las estimaciones de TTT en t\'erminos de la media y la varianaza de las estimaciones de AAGo (tomado solo la primera partida de cada jugador).

 \begin{figure}[H]\centering
\begin{subfigure}[t]{0.48\textwidth}
\includegraphics[width=\textwidth]{../../figures/pdf/aago_ttt_escalado} 
\caption{Trueskill Through Time (TTT)}
\label{aago_ttt_escalado}
\end{subfigure}
\begin{subfigure}[t]{0.48\textwidth}
\includegraphics[width=\textwidth]{../../figures/pdf/aago_aga} 
\caption{RAAGO (ranking)}
\label{aago_aga}
\end{subfigure}
\caption{Funci\'on de mapeo $x\text{d} = x$ y $x\text{k} = -x + 1$}
\end{figure}

Una comparaci\'on correcta de las algoritmos de estimaciones de habilidad requiere contar con las predicciones a priori de los resultados observados.
Solo a modo exploratorio, mostramos la diferencia de ranking entre estimaciones (Fig~\ref{aago_diferencia_escalada}). 

\vspace{0.3cm}

\textbf{Nota}: La cantidad de estiamaciones disponibles en la tabla \texttt{events\_eventplayer}, no coinciden de forma perfecta con la cantidad de estimaciones que surgen usando la tabla \texttt{games\_game} filtrando en base a las recomendaciones enviadas por mail 66 partidas por razones de \texttt{Result: 3, Reaso: 55, Unrated: 10}. En el siguiente gr\'afico se descartan las estimaciones AAGo que no tienen correlato en Trueskill.

 \begin{figure}[H]\centering
\begin{subfigure}[t]{0.66\textwidth}
\includegraphics[width=\textwidth]{../../figures/pdf/aago_diferencia_escalada} 
\end{subfigure}
\caption{}
\label{aago_diferencia_escalada}
\end{figure}

Este gr\'afico parece indicar que existe una diferencia significativa entre las estimaciones que surgen de uno y otro algoritmos.
Otra diferencia significativa surge de comparar la selecci\'on de handicap que hace uno y otro algoritmo (Fig~\ref{aago_handicap}). 

\begin{figure}[H]\centering
\begin{subfigure}[t]{0.66\textwidth}
\includegraphics[width=\textwidth]{../../figures/pdf/aago_handicap} 
\end{subfigure}
\caption{Rosa: AAGo. Celeste: TTT}
\label{aago_handicap}
\end{figure}

Del conjunto de partidas que tuvieron asignado handicap el $61.3\%$ de las veces gana el blanco.
Esto indica que el handicap no alcanza (por el motivo que sea) para hacer que las partidas sean parejas.

\vspace{0.3cm}

Estas difrencias no dicen nada respecto de qu\'e algoritmo funciona mejor.
Para evaluarlos es necesario contar con las predicciones a priori de RAAGo.



\end{document}
