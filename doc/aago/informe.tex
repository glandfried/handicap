\documentclass[a4paper,10pt]{report}
\usepackage[utf8]{inputenc}
\usepackage[spanish,es-tabla]{babel}
\usepackage{authblk}
\usepackage{cite}
\usepackage{framed}
\usepackage{color}
\usepackage{graphicx}
\usepackage{parskip}
\usepackage{enumerate}
\usepackage{fullpage}
\usepackage[colorinlistoftodos, textsize=small]{todonotes}
\usepackage{wrapfig}
\usepackage{multirow}
\usepackage{comment}
\usepackage{tikz}
\tikzstyle{arrow}=[draw, -latex]
\input{tikzlibrarybayesnet.code.tex}
\usepackage{array}    % para que tabular funcione centrado
\usepackage{url}
\usepackage{amsmath}
\usepackage{caption}
\usepackage{subcaption}

% El paquete authblk no soporta español, no traduce el "and" que separa los autores
\renewcommand\Authand{, y }
\renewcommand\Authands{, y }

\title{Estimaci\'on de habilidad en Go}
\author[a]{Martín Amigo}
\author[a]{Tobías Carreira Munich}
\author[a]{\\ \vspace{0.3cm} \normalsize Directores: Esteban Mocskos y Gustavo Landfried}
% \author[a]{Codirector: Gustavo Landfried}

\date{\today}

\affil[a]{\small Universidad de Buenos Aires. Facultad de Ciencias Exactas y Naturales. Departamento de Computaci\'on. Buenos Aires, Argentina}

\begin{document}

\maketitle

\section*{Metodología}

\section*{AGA/AAGo}
\begin{itemize}
  \item vimos que la evidencia daba muy bajita, casi cero (poner numerito, calcular la online también)
  \item como se calcula multiplicativamente, podía ser que alguna o algunas partidas estén dando muy bajo y arrastrando error
  \item pensamos que quizás eran las partidas iniciales de los jugadores, donde no hay prior. pero las sacamos y no cambiaba mucho (cuantificar esto). tiene sentido que no cambie mucho porque son tipo 100 partidas de 3300
  \item buscamos cuántas partidas daban muy bajo. eran muchas. hicimos un histograma. aparte de muchas altas, hay muchisimas bajas, muchas bajísimas y bastantes muuuy bajisisimas.
  \item nos fijamos la relacion con handicap: claramente es eso. cuando el handicap es bastante alto las predicciones son malísimas. [intentar explicar por qué?]
\end{itemize}}



\section*{Whole History Rating}

\section*{TrueSkill Through Time}


% TODO: bibliografia

\end{document}
