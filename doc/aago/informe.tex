\documentclass[a4paper,10pt]{report}
\usepackage[utf8]{inputenc}
\usepackage[spanish,es-tabla]{babel}
\usepackage{authblk}
\usepackage{cite}
\usepackage{framed}
\usepackage{color}
\usepackage{graphicx}
\usepackage{parskip}
\usepackage{enumerate}
\usepackage{fullpage}
\usepackage[colorinlistoftodos, textsize=small]{todonotes}
\usepackage{wrapfig}
\usepackage{multirow}
\usepackage{comment}
\usepackage{tikz}
\tikzstyle{arrow}=[draw, -latex]
\input{tikzlibrarybayesnet.code.tex}
\usepackage{array}    % para que tabular funcione centrado
\usepackage{url}
\usepackage{amsmath}
\usepackage{caption}
\usepackage{subcaption}

% El paquete authblk no soporta español, no traduce el "and" que separa los autores
\renewcommand\Authand{, y }
\renewcommand\Authands{, y }

\title{Estimaci\'on de habilidad en Go}
\author[a]{Martín Amigo}
\author[a]{Tobías Carreira Munich}
\author[a]{\\ \vspace{0.3cm} \normalsize Directores: Esteban Mocskos y Gustavo Landfried}
% \author[a]{Codirector: Gustavo Landfried}

\date{\today}

\affil[a]{\small Universidad de Buenos Aires. Facultad de Ciencias Exactas y Naturales. Departamento de Computaci\'on. Buenos Aires, Argentina}

\begin{document}

\maketitle

\section*{Metodología}

En este trabajo nos ocupamos de revisar distintos modelos matemático computacionales para estimar la habilidad de jugadores de Go.
Utilizaremos los principales métodos de la bibliografía, teniendo como base el que usa actualmente la Asociación Argentina de Go (AAGo).
Además, propondremos extensiones a uno de los métodos (TrueSkill Through Time) para que contemple tanto el handicap como el komi de cada partida.

\subsection*{Datos}

Trabajamos con el dataset provisto por la AAGo, del cual nos quedamos con las partidas válidas rankeadas. Cuenta con cinco partidas durante el año 2012, y 3313 partidas entre el 27 de junio del 2016 y el 26 de abril del 2020. \todo{Describir un poco más los datos? Cantidad de jugadores, gráficos generales?}

\subsection*{Comparación de modelos}

Al trabajar con distintos modelos, vamos a querer compararlos para saber cual se aproxima mejor a la realidad.
En este cap\'itulo describiremos la metodolog\'ia bayesiana basada en probabilidad, usando la evidencia de los modelos.
Esta forma introduce un factor que penaliza modelos m\'as complejos de lo necesario, mediante un t\'ermino conocido como Factor de Ockham, que surge naturalmente de las reglas de la probabilidad.
Esto evita sobre ajustes, sin necesidad de acudir a penalizadores ad-hoc.

Inicialmente, vamos a considerar dos niveles de inferencia:

\begin{itemize}
	\item En el primer nivel, asumimos que el modelo con el que trabajamos es el verdadero, y ajustamos nuestro modelo a los datos
	\item En el segundo nivel, tenemos la comparaci\'on de modelos, en la cu\'al usamos la evidencia del primer nivel
\end{itemize}

En el primer nivel, suele haber un espacio de hip\'otesis sobre el cual queremos aprender utilizando los datos.
Por la regla de Bayes:

\begin{equation}\label{eq:bayes-parametros-modelo}
	P(\theta | D, M) = \frac{P(D|\theta, M) P(\theta|M)}{\int P(D|\theta, M) P(\theta|M) dw} = \frac{P(D|\theta, M) P(\theta|M)}{P(D|M)}
\end{equation}

donde $\theta$ son las hip\'otesis del modelo, $M$ es el modelo y $D$ son los datos.
Notamos que en la ecuación \ref{eq:bayes-parametros-modelo}, la evidencia es la \textbf{probabilidad de los datos, dado que el modelo $M$ es el verdadero}: $P(D|M)$.

El segundo nivel de inferencia utiliza la evidencia del primer nivel para calcular la probabilidad del modelo, dados los datos:

\begin{equation}\label{eq:bayes-modelos}
	P(M | D) \propto P(D|M) P(M)
\end{equation}

En el caso de que no tengamos informaci\'on previa que nos haga preferir un modelo por sobre el resto, $P(M)$ ser\'a equiprobable, por lo que elegir el modelo con mayor probabilidad equivale a quedarse con el modelo de mayor evidencia.
De aqu\'i podemos sacar el concepto de Factor Bayesiano\cite{kass1995-bayesFactors}:

\begin{equation}
	\frac{P(D|M_i)}{P(D|M_j)}
\end{equation}

que resume en un n\'umero la evidencia provista por los datos a favor de un modelo o teor\'ia por sobre otra.

Por la regla del producto, podemos descomponer la evidencia en la probabilidad de cada dato, dado el modelo y los datos anteriores (ecuación \ref{eq:productoria-evidencia}).
En nuestro contexto, esto se traduce en calcular la probabilidad de cada partida, dado que el método conoce el resultado de las partidas anteriores.
Es destacable el uso de la productoria, ya que una sola partida con probabilidad 0 provoca que la evidencia sea 0.

\begin{equation}\label{eq:productoria-evidencia}
	P(D|M) = \prod P(d_i|M, d_{1}, .., d_{i-1})
\end{equation}

Como la evidencia es un número pequeño y poco interpretable, en múltiples ocasiones usaremos el promedio geométrico de las evidencias individuales de las partidas.

\section*{AGA/AAGo}
\begin{itemize}
  \item vimos que la evidencia daba muy bajita, casi cero (poner numerito, calcular la online también)
  \item como se calcula multiplicativamente, podía ser que alguna o algunas partidas estén dando muy bajo y arrastrando error
  \item pensamos que quizás eran las partidas iniciales de los jugadores, donde no hay prior. pero las sacamos y no cambiaba mucho (cuantificar esto). tiene sentido que no cambie mucho porque son tipo 100 partidas de 3300
  \item buscamos cuántas partidas daban muy bajo. eran muchas. hicimos un histograma. aparte de muchas altas, hay muchisimas bajas, muchas bajísimas y bastantes muuuy bajisisimas.
  \item nos fijamos la relacion con handicap: claramente es eso. cuando el handicap es bastante alto las predicciones son malísimas. [intentar explicar por qué?]
\end{itemize}}



\section*{Whole History Rating}

\section*{TrueSkill Through Time}


% TODO: bibliografia

\end{document}
