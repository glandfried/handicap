\documentclass[shownotes]{beamer}

\mode<presentation>
{
%   \usetheme{Madrid}      % or try Darmstadt, Madrid, Warsaw, ...
%   \usecolortheme{default} % or try albatross, beaver, crane, ...
%   \usefonttheme{default}  % or try serif, structurebold, ...
 \usetheme{Antibes}
 \usecolortheme[rgb={0.6,0.75,0}]{structure}%divido los RGB por 252
 \setbeamercolor{block title}{fg=white,bg=azuluca}
 \xdefinecolor{azuluca}{rgb}{0.02, 0.2, 0.18}
 \definecolor{greenblue}{rgb}{0.1, 0.55, 0.5}

 \setbeamercolor{palette quaternary}{fg=white,bg=azuluca}
 \setbeamertemplate{caption}[numbered]
 \setbeamertemplate{navigation symbols}{}

}

\input{./aux/tex/diapo_encabezado.tex}
%\input{./aux/tex/tikzlibrarybayesnet.code.tex}

\title[Handicap]{Handicap}

\author[Gustavo Landfried]{Gustavo Landfried \\ \vspace{0.2cm}
\scriptsize Lic. Ciencias Antropol\'ogicas. \\
Doctorando Ciencias de la Computaci\'on \\
\vspace{-0.3cm}}
\institute[DC-ICC-CONICET]{Departamento de Ciencias de la Computaci\'on (UBA -- CONICET) \vspace{-0.3cm}}
\date{}


\begin{document}
\small 

\begin{frame}[noframenumbering]

\maketitle

 \begin{textblock}{80}(72,82)
 \includegraphics[width=0.2\textwidth]{aux/images/logo_licar} 
 \end{textblock}
  \begin{textblock}{80}(42,82)
 \includegraphics[width=0.2\textwidth]{aux/images/logo_version_02} 
\end{textblock}

\end{frame}

\section{Problema}

\begin{frame}
  Que es el handicap, ejemplos de uso.
\end{frame}


\begin{frame}
 
 \begin{framed} \centering
  Efecto de handicap en la estimaci\'on de habilidad: caso GO.
 \end{framed}
 
\end{frame}

\begin{frame}
 \begin{framed} \centering
  ¿Son correctas las heuristicas para otorgar 
  
  ventajas (handicap) en el juego Go?
 \end{framed}
 
\end{frame}


\section{Metodolog\'ia}
\begin{frame}
 
\only<1>{
\begin{textblock}{128}(0,8)
\begin{center}
 \large TrueSkill
\end{center}
\end{textblock}
\begin{textblock}{128}(0,33)
\tikz{ %
        
        \node[det, fill=black!10] (r) {$r_{ij}$} ; %
        \node[const, left=of r, xshift=-1.4cm] (r_name) {\small Resultado observado}; 
        \node[const, right=of r] (dr) {\large $ r_{ij} = \mathbb{I}(p_i>p_j)$}; 
          
         \node[latent, above=of r, xshift=-0.8cm] (p1) {$p_i$} ; %
         \node[latent, above=of r, xshift=0.8cm] (p2) {$p_j$} ; %
         \node[const, left=of p1, xshift=-0.55cm] (p_name) {\small Rendimiento oculto}; 
         
         \node[latent, above=of p1] (s1) {$s_i$} ; %
         \node[latent, above=of p2] (s2) {$s_j$} ; %
         %\node[latent, above=of p2,xshift=1.2cm] (h) {$h_n$} ; %
                  
         \node[const, right=of p2] (dp2) {\large $p \sim N(s,\beta^2)$};
         
         
%          \node[latent, above=of p1, fill=black, minimum size=1pt] (s1) {} ; %
%          \node[latent, above=of p2, fill=black, minimum size=1pt] (s2) {} ; %
%          \node[const, above=of s1] (ds1) {\large $s_i$};
%          \node[const, above=of s2] (ds2) {\large $s_j$};
          \node[const, left=of s1, xshift=-.85cm] (s_name) {\small Habilidad oculta}; 
%          
         \edge {p1,p2} {r};
         \edge {s1} {p1};
         \edge {s2} {p2};
         
         \node[invisible, right=of p2, xshift=4.75cm] (s-dist) {};
} 
\end{textblock}
}
\only<2>{
\begin{textblock}{128}(0,8)
\begin{center}
 \large TrueHandicap
\end{center}
\end{textblock}
\begin{textblock}{128}(0,33)
\tikz{ %
        
        \node[det, fill=black!10] (r) {$r_{ij}$} ; %
        \node[const, left=of r, xshift=-1.4cm] (r_name) {\small Resultado observado}; 
        \node[const, right=of r] (dr) {\large $ r_{ij} = \mathbb{I}(p_i>p_j)$}; 
          
         \node[latent, above=of r, xshift=-0.8cm] (p1) {$p_i$} ; %
         \node[latent, above=of r, xshift=0.8cm] (p2) {$p_j$} ; %
         \node[const, left=of p1, xshift=-0.55cm] (p_name) {\small Rendimiento oculto}; 
         
         \node[latent, above=of p1] (s1) {$s_i$} ; %
         \node[latent, above=of p2] (s2) {$s_j$} ; %
         \node[latent, above=of p2,xshift=1.2cm] (h) {$h_n$} ; %
                  
         \node[const, right=of p2] (dp2) {\large $p \sim N(s + h,\beta^2)$};
         
         
%          \node[latent, above=of p1, fill=black, minimum size=1pt] (s1) {} ; %
%          \node[latent, above=of p2, fill=black, minimum size=1pt] (s2) {} ; %
%          \node[const, above=of s1] (ds1) {\large $s_i$};
%          \node[const, above=of s2] (ds2) {\large $s_j$};
          \node[const, left=of s1, xshift=-.85cm] (s_name) {\small Habilidad oculta}; 
%          
         \edge {p1,p2} {r};
         \edge {s1} {p1};
         \edge {s2,h} {p2};
         
         
         \node[invisible, right=of p2, xshift=4.75cm] (s-dist) {};
} 
\end{textblock}
}


\end{frame}

\section{Resultados}

\begin{frame}
  \begin{figure}[H]     
     \centering \normalsize
     \begin{subfigure}[b]{0.8\textwidth}
     \includegraphics[width=1\textwidth]{../figures/pdf/handicap19.pdf} 
     \end{subfigure}
  \end{figure}
\end{frame}

\begin{frame}
  \begin{figure}[H]     
     \centering \normalsize
     \begin{subfigure}[b]{0.8\textwidth}
     \includegraphics[width=1\textwidth]{../figures/pdf/handicap19_history.pdf} 
     \end{subfigure}
  \end{figure}
\end{frame}

\begin{frame}
  \begin{figure}[H]     
     \centering \normalsize
     \begin{subfigure}[b]{0.8\textwidth}
     \includegraphics[width=1\textwidth]{../figures/pdf/handicap19_population.pdf} 
     \end{subfigure}
  \end{figure}
\end{frame}


\begin{frame}
  \begin{figure}[H]     
     \centering \normalsize
     \begin{subfigure}[b]{0.8\textwidth}
     \includegraphics[width=1\textwidth]{../figures/pdf/handicap19_population_distribution.pdf} 
     \end{subfigure}
  \end{figure}
\end{frame}

\begin{frame}
  \begin{figure}[H]     
     \centering \normalsize
     \begin{subfigure}[b]{0.8\textwidth}
     \includegraphics[width=1\textwidth]{../figures/pdf/handicap9.pdf} 
     \end{subfigure}
  \end{figure}
\end{frame}

\begin{frame}
  \begin{figure}[H]     
     \centering \normalsize
     \begin{subfigure}[b]{0.8\textwidth}
     \includegraphics[width=1\textwidth]{../figures/pdf/handicap9_history.pdf} 
     \end{subfigure}
  \end{figure}
\end{frame}

\begin{frame}
  \begin{figure}[H]     
     \centering \normalsize
     \begin{subfigure}[b]{0.8\textwidth}
     \includegraphics[width=1\textwidth]{../figures/pdf/handicap9_population.pdf} 
     \end{subfigure}
  \end{figure}
\end{frame}


\begin{frame}
  \begin{figure}[H]     
     \centering \normalsize
     \begin{subfigure}[b]{0.8\textwidth}
     \includegraphics[width=1\textwidth]{../figures/pdf/handicap13.pdf} 
     \end{subfigure}
  \end{figure}
\end{frame}

\begin{frame}
  \begin{figure}[H]     
     \centering \normalsize
     \begin{subfigure}[b]{0.8\textwidth}
     \includegraphics[width=1\textwidth]{../figures/pdf/handicap13_history.pdf} 
     \end{subfigure}
  \end{figure}
\end{frame}

\begin{frame}
  \begin{figure}[H]     
     \centering \normalsize
     \begin{subfigure}[b]{0.8\textwidth}
     \includegraphics[width=1\textwidth]{../figures/pdf/handicap13_population.pdf} 
     \end{subfigure}
  \end{figure}
\end{frame}


\section{Gracias}

\begin{frame}
 
 \begin{center}
  \Large Gracias
 \end{center}

 \begin{figure}[H]     
     \centering
     \begin{subfigure}[b]{0.45\textwidth}
       \includegraphics[width=\textwidth]{aux/images/pachacuteckoricancha.jpg} 
     \end{subfigure}
   \end{figure} 
  
\end{frame}


\end{document}



